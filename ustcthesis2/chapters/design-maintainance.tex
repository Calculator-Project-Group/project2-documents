\chapter{维护设计}
%可能的内容包括数据库的日常备份、压缩、维护等。

\section{数据库维护}
\begin{itemize}
    \item 定期对数据库状态巡检。
    \item 定期对数据库进行优化和垃圾清理。
    \item 做好数据库备份情况与可恢复性检查。
    \item 数据库检查的操作必须在业务相对比较空闲的时候执行,否则可能会影响系统性能。
    \end{itemize}

\section{日志管理}
服务器在运行时会产生大量日志,管理员应当定期查看和归档日志。
在服务上线初期,可以将日志记录级别调低,以便发现问题。
待服务运行稳定后,则可以调高日志记录级别,提高性能。


\section{邮件提醒}
当系统内部出现错误时,会发送邮件到管理员邮箱。
管理员可以在web后台调整提醒等级,决定在什么级别的错误发生时才发送邮件。
