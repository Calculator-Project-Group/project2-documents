\chapter{总体设计}
\section{软件描述}
%系统包括前台和后台两个部分。

%前台主要功能是:

%后台主要功能是:
系统包括客户端和服务端两个部分。

客户端为用户提供图形界面,用户可以在客户端上登陆账号,搜索添加好友,和好友聊天,参与群聊,发送语音视频消息。

服务端为客户端提供支持,存储用户数据和消息记录,在不同用户间转发聊天消息,提供搜索功能。

\section{处理流程}
\subsection{总体流程}
%此处应当有一个图和对应的描述。

\begin{figure}[h]
	\centering
	\includegraphics[width=13cm]{base_flowchart}
	\caption{总体流程图} \label{fig:base_flowchart}
\end{figure}

%\subsection{系统基本流程}
%此处应当有一个图和对应的描述。

\subsection{客户端基本流程}
%这只是举个例子,如果没有客户端则不需要此节。
\begin{figure}[h]
	\centering
	\includegraphics[width=13cm]{client_flowchart}
	\caption{客户端流程图} \label{fig:client_flowchart}
\end{figure}
用户在登陆账号后,客户端会从服务器同步好友,群组和会话信息。并打开一个到服务器的长连接,监听服务器的推送消息。
用户可以点击好友列表或群组列表中的项目启动一个会话窗口,在会话窗口中发送各种消息。
服务器会及时推送消息给客户端,客户端会将消息及时显示到对应的窗口中。
客户端也提供好友搜索,个人资料修改等功能。
见图-\ref{fig:client_flowchart}

\subsection{服务器端基本流程}
%这只是举个例子,如果没有服务器端则不需要此节。
\begin{figure}[h]
	\centering
	\includegraphics[width=13cm]{server_flowchart}
	\caption{服务端流程图} \label{fig:server_flowchart}
\end{figure}

服务端启动后连接数据库,监听端口,响应来自客户端的请求。
处理注册请求,验证合法后向数据库插入一条新纪录。
处理登陆请求,验证合法后为该账号分配一个 session 对象,
并维持一条到客户端的长连接,聊天消息,好友请求都会从该连接推送给客户端。
服务端亦提供 http api,用于修改个人资料,查找用户等功能。
见图-\ref{fig:server_flowchart}


\subsection{登录功能具体流程}
%此处应当有描述。
\begin{figure}[h]
	\centering
	\includegraphics[width=5cm]{login_flowchart}
	\caption{登录功能流程图} \label{fig:login_flowchart}
\end{figure}
用户打开客户端,输入用户名和密码,点击登录提交至服务器,服务器查询数据库验证用户名密码是否合法。
若合法则进入主界面,否则提示用户出错。
见图-\ref{fig:login_flowchart}


\subsection{用户搜索功能具体流程}
%此处应当有一个描述。
\begin{figure}[h]
	\centering
	\includegraphics[width=13cm]{search_friend_flowchart}
	\caption{用户搜索功能流程图} \label{fig:search_friend_flowchart}
\end{figure}
用户打开搜索窗口,在搜索框输入要搜索的内容,一般是用户名的部分匹配。点击按钮提交给服务器。
服务器获取参数后查询数据库,获取返回结果,若无记录,则返回错误代码,客户端收到响应后显示无结果。
否则客户端将收到的用户数据以列表形式显示出来。
用户可以点击搜索结果中的项目打开用户信息窗口,添加好友。

\subsection{聊天功能具体流程}
用户在客户端的好友列表或群组列表选择项目,打开会话窗口,即可发送文本,图片或者语音消息。
消息数据会通过长连接发送给服务器。
服务器在接收到消息后就会推送给相应的用户。如果是群聊,则广播给所有群成员。
其他客户端收到后,若之前并未打开会话窗口,则会收到消息提醒。
若会话窗口已打开,则消息内容会显示到会话窗口中。
消息本身也将保存在服务器的数据库中。
不在线的用户上线后,服务器会推送未读的消息给客户端。

\section{功能结构设计}
\subsection{整体结构}
%此处应当有一个图和对应的描述。系统如果像微内核那样,划分成核心模块和若干个子系统,此处应当有图示及说明,然后后续几个节应当描述这几个子系统。如果系统像宏内核,那应当说明有哪些紧密联系的模块,并在后续几个节内描述这些模块。
产品的整体结构如图-\ref{fig:overall_architecture}所示。
\begin{figure}[]
	\centering
	\includegraphics[width=13cm]{overall_architecture}
	\caption{整体结构} \label{fig:overall_architecture}
\end{figure}

%\subsection{用户端结构}
% 此处应当有一个图和对应的描述。这只是举个例子。可能的内容包括用户端的具体模块、耦合情况等。
% 用户端结构如图-\ref{fig:}

\subsection{服务器端结构}
% 此处应当有一个图和对应的描述。这只是举个例子。
服务器端各模块的结构图如图-\ref{fig:backend_architecture}所示。
\begin{figure}[]
	\centering
	\includegraphics[width=13cm]{backend_arch}
	\caption{服务器端结构} \label{fig:backend_architecture}
\end{figure}
\subsubsection{Django Web应用与Python运行时}
该模块是本Web系统的最重要的成分之一。作为后端,它包含了产品大部分的业务逻辑。

Django Web应用内部,还大致分为Request Dispatcher和Views两大模块。Request Dispatcher接收由代理转发来的HTTP请求,根据其地址,将其传入不同的API函数。

包含API内部函数会根据请求,做必要的数据处理和数据库操作,再生成HTTP响应返回给用户。

\subsubsection{Nginx代理}
该模块将作为前端服务器程序,处理所有到达服务器的请求(包括对静态资源的请求)。同时为Django应用提供反向代理。

\subsubsection{HTML/CSS模板和JS脚本}
为Web客户端界面提供模板、样式和必要的动画。由Nginx模块和Django Web应用读取,并生成HTTP应答返回给用户。

\subsubsection{MySQL数据库}
该模块存储所有的用户、会话及其他数据。数据库会与Django Web应用以\href{https://www.python.org/dev/peps/pep-0249}{PEP 249}所定义的API进行通信。



% \subsection{后台数据库维护模块结构}
% 此处应当有一个图和对应的描述。这只是举个例子。

{\color{red}
\section{消息时序的一致性设计}
多用户聊天时,由于网络时延的不确定性,原本在一台客户端顺序发送的消息,可能会在服务器或者另一台客户端乱序到达。本节明确了服务器和客户端显示接受到的消息时维护其时序一致性的处理规范。\\


\subsection{双人聊天}
双人聊天需要保证发送方的发送顺序与接受方所看到的是一致的。

故发送方在发送每一条消息时,不仅需要为其附上时间戳,还需要为其加上递增的序列号。
若服务器收到的消息的序列号出现不连续,则在丢失的消息被收到前,后续的消息不会被推送给接收方的客户端。

\textbf{工作流程示例:}

单人聊天,发送方A依次发出了msg1,msg2,msg3三个消息给接收方B,则整个工作流程如下:
\begin{enumerate}
	\item A 发出消息 msg1,msg2,msg3
	\item 因为网络状况的问题,消息msg3 先到达服务器
	\item 服务器根据消息的序列号,知晓前面还有两条消息,阻塞等待
	\item 消息msg1, msg2 到达服务器
	\item 通过推送服务将消息按顺序发给 B
\end{enumerate}

同样,阻塞动作以及相应的排序工作也可由接收方客户端完成,此时服务器只要收到消息就转发。
客户端负责维护接收队列并排序,将消息按顺序显示给用户。
这样可以减轻服务器压力,同时避免服务器推送消息到客户端时出现的时序性问题。

\subsection{多人聊天}
多人聊天中,由于不同客户端发出的消息的发送时间可能存在不统一的误差(如客户端的系统时间未校准),故不同客户端的消息应统一按照服务器接收到的时间进行排序。
多人群聊,需要保证每个用户看到的消息顺序都是一致的(所有接收方展现顺序一致)。

具体做法为在服务器端维护消息的计数器(也可不维护,直接由接收时间排序)。
% 其实服务器端可以不用称作计数器了,因为就是按照接受时间排列

\textbf{工作流程示例:}

\begin{enumerate}
	\item A 发出消息msg1,B 发出消息msg2
	\item 消息msg1 和 msg2 分别到达服务器
	\item 服务器维护一个唯一的序列号,来确定接收方展示时序
	\item 服务器分配给消息msg1的序列号是20,消息msg2的序列号是21
	\item 通过推送服务将消息发给多个群友,群友即使接收到消息msg1和msg2的时间不同,但可以统一按照序列号来展现。
\end{enumerate}
该设计存在问题是,无法保证同一个用户的消息按发送顺序来展示。后续可考虑通过同时维护客户端的局部序列号与服务器端的全局序列号,进行多重排序来解决。
}

\section{功能需求与程序代码的关系}
% [此处指的是不同的需求分配到哪些模块去实现。可按不同的端拆分此表]
\begin{table}[htbp]
\centering
\caption{功能需求与程序代码的关系表} \label{tab:requirement-module}
\begin{tabular}{|c|c|c|c|}
    \hline
    需求名称  & 对应模块 \\
    \hline
	R.INSTANT.MESSAGE.USERS.001 & · \\
	新用户注册 & 数据库、Views模块内部的注册子模块 \\
    \hline
	R.INSTANT.MESSAGE.USERS.002 & · \\
	变更自己的用户资料 & 数据库、Views模块内部的用户子模块 \\
    \hline
	R.INSTANT.MESSAGE.USERS.003 & · \\
	查找用户资料 & 数据库、Views模块内部的用户子模块\\
	\hline
	R.INSTANT.MESSAGE.USERS.004 & · \\
	申请好友 & 数据库、Views模块内部的用户子模块 \\
	\hline
	R.INSTANT.MESSAGE.CHAT.001/002/003 & · \\
	发送文本、语音、图片消息 & 数据库、Views模块内部的消息子模块 \\
	\hline
	R.INSTANT.MESSAGE.CHAT.008 & · \\
	视频聊天 & 数据库、Views模块内部的数据流子模块 \\
	\hline
	R.INSTANT.MESSAGE.CHAT.004 & · \\
	创建聊天群组 & 数据库、Views模块内部的群组子模块 \\
    \hline
	R.INSTANT.MESSAGE.CHAT.005 & · \\
	邀请用户加入聊天群组 & 数据库、Views模块内部的群组子模块 \\
	\hline
	R.INSTANT.MESSAGE.CHAT.006/007 & · \\
	退出、解散聊天群组 & 数据库、Views模块内部的群组子模块 \\
	\hline
	\hline
\end{tabular}
\note{各项功能需求的实现与各个程序模块的分配关系}
\end{table}