\chapter{数据库设计}
\section{数据库环境说明}
%本系统的数据系统采用MySQL/PostgreSQL/Microsoft SQL Server数据库系统。

%其中xxx模块因为xxx而需要用到Hadoop架构。

本系统的数据系统采用MySQL数据库系统。

\section{数据库的命名规则}
%是否允许单词缩写,允许的单词缩写有哪些。
允许单词缩写
identity - ID
password - pw

表名是单数还是复数。关联表如何命名。字符数限制等。

% 字段是否带上前缀(如integer类型则加上i前缀等)。不带

\section{逻辑设计}
%是否需要满足某一种范式。
逻辑设计满足 BCNF

画个实体的逻辑关系表/图在此处。

\section{物理设计}
\subsection{数据库产品}
%用哪家数据库,是否分布式等。
mysql

\subsection{实体属性、类型、精度}
\subsubsection{客户数据表设计}
\begin{table}[htbp]
\centering
\caption{用户数据表Users设计} \label{tab:client-database}
\begin{tabular}{|c|c|c|c|c|}
    \hline
    字段名 & 类型 & 大小 & 说明 & 备注 \\
    \hline
    ID & char & 64 & 用户的唯一标识符 & 主键 \\
    \hline
    pw & char & 512 & 用户的登录密码hash值 & · \\
    \hline
    email & char & 64 & 用户的注册邮箱 & 唯一 \\
    \hline
    last\_login & time & 16 & 最近登录时间 & · \\
    \hline
    register\_time & time & 16 & 注册时间 & · \\
    \hline
    birthday & date & 16 & 生日 & · \\
    \hline
    sex & char & 4 & 性别 & · \\
    \hline
\end{tabular}
\note{用户数据表Users设计}
\end{table}

\subsubsection{订单数据表设计}
\begin{table}[htbp]
\centering
\caption{订单数据表Orders设计} \label{tab:order-database}
\begin{tabular}{|c|c|c|c|c|}
    \hline
    字段名 & 类型 & 大小 & 说明 & 备注 \\
    \hline
    ID & char & 64 & 订单的唯一标识符 & 主键\\
    \hline
    user & char & 64 & 对应用户 & 外键,来自xx表 \\
    \hline
\end{tabular}
\note{订单数据表Orders设计}
\end{table}
\section{安全性设计}
%备份和容灾设计。
本产品的服务端主要面向有自己搭建即时通讯服务需求的个人用户和中小企业用户,
预期使用规模较小,所以没有特别的容灾设计,
数据备份工作依靠服务端维护人员自行开展。

\section{数据库管理与维护说明}
对于数据库的维护,随时对数据库中的信息加以调试和保存备份。同样需要个工作人员进行系统的分析和用户的反馈,对系统进行升级以及功能的完善。同时保证系统安全有序的运行。