\chapter{数据结构设计}
\section{逻辑结构设计}
% \subsection{用户管理系统数据结构设计}
%讲述本系统内需要什么数据结构。这指的是程序运行过程中维护的数据结构。只是举个例子,此处应和3.3一致。

\subsection{客户端数据结构}

\subsubsection{好友列表}
客户端需要从服务器拉取自己的好友并维持在本地,并记录好友的在线状态。
用户界面使用该数据结构填充好友列表控件。

该数据结构需要实时与服务端数据库保持同步。

\subsubsection{群组列表}
客户端需要从服务器拉取自己的群组并维持在本地,
用户界面使用该数据结构填充群组列表控件。

该数据结构需要实时与服务端数据库保持同步。

\subsubsection{消息记录}
用户每一个的群组和好友都会有一个消息记录,它顺序记录了一段时间内的所有聊天消息。
用户界面使用该数据结构填充与每个用户的聊天消息框控件。

该数据结构需要实时与服务端数据库保持同步。

\subsubsection{会话记录}
当用户向一个好友或群组发起聊天时,客户端会为该聊天创建一个会话记录对象(如果当时不存在的话)。
该对象包含对一个该聊天所对应的消息记录的引用,以便用户想要继续该会话时,能够直接恢复之前的消息记录(即聊天记录)。
用户界面左侧的会话记录组件使用该数据结构填充。

\subsubsection{输入缓冲}
对应用户未发送的文本、语音、图片消息。

\subsection{服务端数据结构}

\subsubsection{会话池}
对每一个客户端都需要维持一个长连接。
使用一个会话池来维护连接对象。

\subsubsection{路由表}
记录了 url 到内部API接口函数的映射


%\section{物理结构设计}
%各数据结构无特殊物理结构要求。(如果有,比如说hadoop等,应当具体说明)

\section{数据结构与程序模块的关系}
[此处指的是不同的数据结构分配到哪些模块去实现。可按不同的端拆分此表]
\begin{table}[htbp]
\centering
\caption{数据结构与程序代码的关系表} \label{tab:datastructure-module}
\begin{tabular}{|c|c|c|c|}
    \hline
    · & 模块1 & 模块2 & 模块3 \\
    \hline
    结构1 & · & Y & · \\
    \hline
    结构2 & · & Y & · \\
    \hline
    结构3 & · & Y & · \\
    \hline
    结构4 & Y & · & · \\
    \hline
    结构5 & · & · & Y \\
    \hline
\end{tabular}
\note{各项数据结构的实现与各个程序模块的分配关系}
\end{table}