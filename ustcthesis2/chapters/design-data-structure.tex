\chapter{数据结构设计}
\section{逻辑结构设计}
% \subsection{用户管理系统数据结构设计}
%讲述本系统内需要什么数据结构。这指的是程序运行过程中维护的数据结构。只是举个例子,此处应和3.3一致。

\subsection{客户端数据结构}

\subsubsection{好友列表}
客户端需要从服务器拉取自己的好友并维持在本地,并记录好友的在线状态。
用户界面使用该数据结构填充好友列表控件。

\subsubsection{群组列表}
客户端需要从服务器拉取自己的群组并维持在本地,
用户界面使用该数据结构填充群组列表控件。

\subsubsection{消息记录}
用户每一个的群组和好友都会有一份消息记录。
用户界面使用该数据结构填充消息框控件。

\subsubsection{输入缓冲区}
对应用户的文本输入

\subsection{服务端数据结构}

\subsubsection{会话池}
对每一个用户都需要维持一个长连接。
使用一个会话池来维护连接对象。

\subsubsection{路由表}
记录了 url 到内部接口的映射


%\section{物理结构设计}
%各数据结构无特殊物理结构要求。(如果有,比如说hadoop等,应当具体说明)

\section{数据结构与程序模块的关系}
[此处指的是不同的数据结构分配到哪些模块去实现。可按不同的端拆分此表]
\begin{table}[htbp]
\centering
\caption{数据结构与程序代码的关系表} \label{tab:datastructure-module}
\begin{tabular}{|c|c|c|c|}
    \hline
    · & 模块1 & 模块2 & 模块3 \\
    \hline
    结构1 & · & Y & · \\
    \hline
    结构2 & · & Y & · \\
    \hline
    结构3 & · & Y & · \\
    \hline
    结构4 & Y & · & · \\
    \hline
    结构5 & · & · & Y \\
    \hline
\end{tabular}
\note{各项数据结构的实现与各个程序模块的分配关系}
\end{table}