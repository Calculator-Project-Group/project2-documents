\chapter{接口设计}
\section{数据类型}
本节描述各API会用到的数据结构类型。所有POST请求的上传内容,以及响应内容均为JSON格式。
\subsection{符号说明}
\begin{itemize}
    \item  *: 必填
    \item  !: 只读(只会出现在响应内容)
    \item  ?: 可选
    \item  \&: 只写(只会出现在请求的上传内容)
\end{itemize}
\subsection{DateTime}
用零时区存储的符合ISO-8601标准的日期时间字符串。如:2018-05-12T07:12:29.181605Z。
\subsection{User}
\begin{lstlisting}[language=C]
{
    id!: Number,
    last_login!: DateTime | null,
    username: String,
    email: Email,
    description: String,
    avatar_url!: URL,
    address: String,
    site_url: URL,
    description: String
}
\end{lstlisting}
\subsection{Message}
用来表示一条消息。此类型可被扩展以支持语音等多媒体消息。
\begin{lstlisting}[language=C]
{
    id!: Number,
    category!: String,
    user_id*: Number, //发送方的用户ID
    message*: String, // 消息内容
    has_read!: Boolean, // 对方是否已读
    created!: DateTime // 发送时间
}
\end{lstlisting}


\section{外部接口}
服务器对外公开的HTTP API遵循RESTful API设计规范。即,对于相同URL的请求,不同的请求方法(如POST, GET, PUT, DELETE等),往往对应着较为统一、类似的语义。

具体而言,标 API<T> <base> 的为类型 T 的 CRUD API, 操作有如下几类:(每个操作均对应到一个HTTP请求方法)
\begin{itemize}
    \item list : GET <base> -> Page<T>
    \item create : POST <base> -> T
    \item retrieve: GET <base>/<id>/ -> T
    \item update : PUT <base>/<id>/ -> T (输入为 T,完整更新对象)
    \item patch : PATCH <base>/<id>/ -> T (输入为 T?,局部更新对象)
    \item delete : DELETE <base>/<id>/
\end{itemize}

本系统主要包含的HTTP API如下:
\subsection{用户相关}
\subsubsection{\textbf{API<User>} /chat/users}
\textbf{请求方法}
\begin{itemize}
    \item GET

        (要求登录)列出符合条件的用户。由搜索功能调用。

        \textbf{查询参数:}
        \begin{itemize}
            \item username: 字符串,表示用户名,或要查询的用户名的部分匹配
        \end{itemize}
        \textbf{返回结果:}
        \begin{itemize}
            \item 200: 操作成功。
            \item 403: 未登录。
        \end{itemize}
    \item POST
        
        创建一个新用户。

        POST数据的格式为User数据类型,需要提供所有必填的数据项。

        \textbf{返回结果:}
        \begin{itemize}
            \item 200: 操作成功,并自动登录。
            \item 404: 当前已有用户登录。
        \end{itemize}

    \item PATCH
        
        (要求登录)更新用户自己的部分信息,

        POST数据的格式为User数据类型的部分数据项。

        \textbf{返回结果:}
        \begin{itemize}
            \item 200: 操作成功。
            \item 403: 未登录,或试图修改其他用户的信息。
        \end{itemize}

\end{itemize}

\subsubsection{\textbf{POST} /chat/users/login}
用户登录。

\textbf{输入:}
\begin{lstlisting}[language=C]
    {
        username*: String,
        password*: String
    }
\end{lstlisting}

\textbf{返回结果:}
\begin{itemize}
    \item 200: 操作成功。
    \item 404: 当前已有用户登录。
    \item 403: 认证失败,即密码错误。
\end{itemize}

\subsubsection{\textbf{GET} /chat/users/logout}
(要求登录)用户登出。
\textbf{返回结果:}
\begin{itemize}
    \item 200: 操作成功。
    \item 403: 未登录。
\end{itemize}

\subsubsection{\textbf{GET} /chat/users/friends}
(要求登录)获取好友列表。

\textbf{查询参数:}
\begin{itemize}
    \item (可选)username: 字符串,表示需要查询的好友的用户名,或其部分匹配。
    \item (可选)group\_id: 分组id,将queryset限定为group\_id所指定的分组内的好友。
\end{itemize}

\textbf{返回结果:}
\begin{itemize}
    \item 200 - 操作成功,并返回由User数据类型组成的列表。
    \item 404 - 未找到给定条件的用户。
    \item 403 - 未登录。
\end{itemize}
    

\subsubsection{\textbf{POST} /chat/users/<username>/follow}
(要求登录)请求添加用户名为username的用户为好友。

\textbf{返回结果:}
\begin{itemize}
    \item 200: 请求发送成功。
    \item 403: 未登录。
    \item 404: 未找到用户名为username的用户。
\end{itemize}

\subsubsection{\textbf{POST} /chat/users/<username>/unfollow}
(要求登录)删除用户名为username的好友。
\textbf{返回结果:}
\begin{itemize}
    \item 200: 操作成功完成。
    \item 403: 未登录。
    \item 404: 未找到用户名为username的用户。
\end{itemize}

\subsection{消息相关}
\subsubsection{\textbf{API<Message>} /chat/messages}
\textbf{请求方法}
\begin{itemize}
    \item GET
    (要求登录)获取所有与好友的消息列表。

    \textbf{查询参数:}
    \begin{itemize}
        \item anchor: DateTime类型,用来指明此时间戳以后的消息需要接收。
    \end{itemize}

    \textbf{返回结果:}
    \begin{itemize}
    \item 200: 操作成功完成,并返回由Message数据类型组成的列表。
    \item 403: 未登录。
    \end{itemize}
    \item POST
    发送一条消息。

    \textbf{输入:}
    \begin{lstlisting}[language=C]
    {
        client_type: String, // 发送方使用的客户端名称
        sender: Number, // 发送方的用户ID
        message_type_name: String, // 消息类型
        content: String, // 消息内容
        send_to: String // 由逗号分隔的接收人ID列表
    }
    \end{lstlisting}
\end{itemize}

\subsubsection{\textbf{GET} /chat/messages/<message_id>[0-9]+}
(要求登录)获取消息ID为message\_id的一条消息。

\textbf{返回结果:}
    \begin{itemize}
    \item 200: 操作成功完成,并返回由Message数据类型组成的列表。
    \item 403: 未登录。
    \end{itemize}

\subsubsection{\textbf{POST} /chat/messages/mark_stream_as_read}
(要求登录)将指定消息设为已读。在用户在客户端查看这条消息时触发。

\textbf{查询参数:}
    \begin{itemize}
        \item msg\_id: Number类型,用来指明消息ID。
    \end{itemize}

\textbf{返回结果:}
    \begin{itemize}
    \item 200: 操作成功完成。
    \item 403: 未登录。
    \item 404: 未找到符合条件的消息。
    \end{itemize}