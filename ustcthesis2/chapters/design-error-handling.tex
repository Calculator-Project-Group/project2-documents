\chapter{出错处理设计}
\section{出错信息}
为客户端、服务端程序所有可能出现的错误建立如下的错误信息表,它包含错误信号代码、错误含义解释和相应的客户端提示。

\begin{tabular}{|c|c|c|}
\hline

错误码&错误含义&客户端提示 \\

\hline

-1&登录或连接超时&网络请求超时,请重试或检查网络是否畅通 \\

\hline

-2&帐号或密码出错&您输入的帐号或密码不正确,请重试 \\

\hline

-3&验证码出错&验证错误,请重新验证 \\

\hline

-4&新密码和确认密码不同&密码和确认密码不一致 \\

\hline

-5&空用户名&用户名不能为空 \\

\hline

-6&手机号错误&手机号输入有误,应为11位数字 \\

\hline

-7&邮箱格式错误&邮箱输入格式有误 \\

\hline

-8&敏感词输入&按照有关部门要求,您的部分输入已被屏蔽 \\

\hline

\end{tabular}


\section{补救措施}
\subsection{数据库出错}
对于此即时通讯软件,数据库主要存储用户信息和各种通讯记录。若是数据库相关软件出错,则回滚软件至稳定可用的版本,再处理软件错误;若是数据库相关硬件如存储介质出错导致数据库信息破坏,则暂停系统服务,使用最近未出错的备份数据,结合所记录的客户端数据流将其同步更新至当前状态再投入使用。

\subsection{服务器出错}
\begin{table}[htbp]
	\centering
	\caption{服务器错误处理设计} \label{tab:server_error_handling_design}
	\begin{tabular}{|c|c|c|}
		\hline
		错误类型      & 输出 \\
		\hline
		不合法的Url请求 & 提示错误,返回404界面 \\
		\hline
		程序出错抛出异常 & 输入Log日志中的错误信息  \\
		\hline
		数据访问量的过大 & 返回请求失败信息 \\
		\hline
    \end{tabular}
\end{table}

\subsection{某模块失效处理}
使用另一个效率稍低的但是稳定的系统,虽然其提供更少的服务,但其也能进行数据库的实时备份。这样在失效模块处理完成后,可以很快的进行数据同步,同时也可以很快的进行原系统的运行。
