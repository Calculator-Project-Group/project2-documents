\chapter{出错处理设计}
\section{出错信息}
建立错误信息表,针对可能出现的错误为其制定错误信号输出、错误含义解释、错误解决方法和相应的客户端提示。


\section{补救措施}
\subsection{数据库出错}
对于此即时通讯软件,数据库主要存储用户信息和各种通讯记录。若是数据库相关软件出错,则回滚软件至稳定可用的版本,再处理软件错误;若是数据库相关硬件如存储介质出错导致数据库信息破坏,则暂停系统服务,使用最近未出错的备份数据,结合所记录的客户端数据流将其同步更新至当前状态再投入使用。

\subsection{服务器出错}
\begin{table}[htbp]
	\centering
	\caption{服务器错误处理设计} \label{tab:server_error_handling_design}
	\begin{tabular}{|c|c|c|}
		\hline
		错误类型      & 输出 \\
		\hline
		不合法的Url请求 & 提示错误,返回404界面 \\
		\hline
		程序出错抛出异常 & 输入Log日志中的错误信息  \\
		\hline
		数据访问量的过大 & 返回请求失败信息 \\
		\hline
    \end{tabular}
\end{table}

\subsection{某模块失效处理}
使用另一个效率稍低的但是稳定的系统,虽然其提供更少的服务,但其也能进行数据库的实时备份。这样在失效模块处理完成后,可以很快的进行数据同步,同时也可以很快的进行原系统的运行。
