\chapter{可行性分析结果}
% Describe the feasibility analysis results on allocated requirements.

% 描述对分配需求的可行性分析结果。
\section{技术可行性分析}
本产品为典型的服务端-客户端模式(Server-client Mode)系统,开发人员所需要用到的技术栈已经相当成熟。故在一般的使用环境下,实现出产品的现有需求并不会呈现出太大难度。然而,考虑到稳定运营之后,用户量可能会大量增长,服务器会面对更多的网络请求流量。如何确保在成本有限的前提下,通过性能调优、负载均衡等手段使服务器能够稳定地长期运行,是本产品以后的运营过程中必将面临的挑战。
\section{费用与效益分析}
\subsection{费用分析}
本产品的费用成本主要来源于:
\begin{itemize}
    \item 软件成本。指开发过程中所用到的各类工具可能需要购买使用权。
    \item 硬件成本。指开发所使用的设备本身,运营所租用的服务器设备本身带来的成本。
    \item 运营成本。指长期运营、维护服务器所带来的成本。
\end{itemize}

本产品所用到的几乎所有软件、工具全为开源产品,并且我们的使用方式均符合它们所对应的开源协议,故软件成本可以忽略不计。\\

对于学生群体,使用各云服务提供商的云主机作为服务器是更合适的选择。市面上主流的云服务器的价格约为每月几十人民币。
若产品只需求在校内部署,还可使用学校所提供的免费的校内云主机。\\

而运营服务器的成本主要来自于维持服务器运行的电费。以性能较好的PC机的功率来计算,该成本显然是可以接受的。
综合考虑以上各种成本,我们认为该产品的开销是较为经济合理的。
\subsection{效益分析}
本产品是非营利性项目,并未带来显著的经济效益,但由于本项目更多是出于服务开源社区和实践锻炼的目的,且作为学生群体并不关心系统经济效益,因此总的效益是较好的。
\section{开发时间分析}
根据以往的开发经验估计,以每天兼职投入开发(约占工作时间的一半)来计算,大约需要一个月来完成开发。