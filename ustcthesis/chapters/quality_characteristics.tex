\chapter{软件质量特性}

% <Specify any additional quality characteristics for the project that will be important to either the customers or the developers. Some to consider are: adaptability, availability, correctness, flexibility, interoperability, maintainability, portability, reliability, reusability, robustness, testability, and usability. Write these to be specific, quantitative, and verifiable when possible. At the least, clarify the relative preferences for various attributes, such as ease of use over ease of learning.

% 详细说明项目任何其他的质量特性。该特性对客户和开发者都非常重要。考虑的方面包括:适应性,可用性,正确性,灵活性,交互工作能力,可维护性,可移植性,可靠性,可重用性,鲁棒性,可测试性和可用性等。定量的详细描述这些特性,尽可能的可验证。对不同属性之间的重要性加以阐述,如:易用性比易学性更重要。

% <Please use the below sub-section for each attributes separately. You can copy the section for additional attributes. >

% 每一个属性单独使用一个小节描述,可根据需要进行增减,如增加可维护性小节等。

\section{扩展性与可维护性}
在开发阶段,应做到高度模块化,尽量使单模块内聚,模块间解耦合。模块依赖于抽象的政策,再辅以各种不同的机制或者说一系列接口,政策的具体实现建立于对各种已有机制的组合使用之上。如此开发并配置软件可以更好地支持软件的进化,模块化与分离的政策与机制有利于在出现新的需求时对软件进行快速精准的更改以适应。

从动态运行角度来说,为增强软件的鲁棒性和智能,软件实现时会有特别的模块监督各种接口异常并详尽记录异常信息,以及采取自适应方式从异常信息中学习,然后反馈关键的信息,和做一定程度的异常处理。对于不同的运行环境或系统约束,本产品会以科学的方式进行大量的适配调试,在适应各种已经环境的同时,也保留能快速扩展适应新环境的能力。

\section{可用性}
该即时通讯系统在使用上几乎没有门槛,其功能已经高度综合,用户不必花大量的时间和精力来学习和使用它,只需要按照UI的排布导引,顺应自己的直觉进入相应的模块,便能对其所提供的功能一目了然。使用相应的功能也不需要用户做太多配置,简简单单的点击就能达到用户的通讯目的和相关的一系列小需求。本产品到用户手中开始使用时,可以基本不需要太多导引就能让用户上手,注册自己的ID,添加目标对象作为自己的好友,和最重要的,编辑信息发送给对象以及自动获取对象的发送信息,由此简单轻松地开始用户与好友的即时交流。

\section{正确性}
软件会经过多环境长时间严格的测试,保障在环境状态允许下,用户与用户之间的交流内容能够准确无误且快速地送达,或是用户上传的信息如个人资料能完成符合用户要求的更改,搜索方面也将以正确的信息回应用户。环境状态不合适时比如网络不佳的状态下通讯,软件会立刻反馈给用户关键的问题提示,使用户知道问题所在并有办法解决问题。

\section{灵活性}
由于该软件的简单可用,这里没有提供脚本控制等一系列复杂扩展功能。主要是实现了部分功能的参数化配置,用户通过简单的配置就能灵活实现不同的复杂功能。同时软件的功能扩展也部分依赖于参数化配置时参数类型的增加。

% \section{交互工作能力}

% \section{可维护性}

% \section{可移植性}

% \section{可靠性}

% \section{可重用性}

% \section{可测试性}

