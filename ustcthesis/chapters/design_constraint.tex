\chapter{总体设计约束}
% <Describe any items or issues that will limit the options available to the developers. >

% 描述可能限制开发人员选择的事项。
 
\section{标准符合性}
< This subsection should specify the requirements derived from existing standards or regulations. In case, if the project refers any International standards, then the deviations from the standards could be specified along with the International standards reference. >

本节详细说明需求所采用的标准或规范的来源。如果项目采用了国际标准,应该说明国际标准及项目与标准的偏离情况。

\section{硬件约束}
<This subsection could include Requirements for the software to operate inside various hardware constraints, such as timing constraints, memory constraints etc.)

本节包括软件在不同的硬件平台运行的需求,如时间相关的约束,内存方面的约束等。

\section{兼容性约束}
为使产品能够在主流的硬件/软件平台上正确地运行,产品需要满足以下约束要求:
\begin{itemize}
    \item Linux平台下的客户端要支持 Ubuntu,Debian,Centos,Arch 等主流发行版
    \item Windows平台下的客户端需支持 Windows 7 及更高
    \item android 平台下的客户端需支持到安卓 4.4 及以上
    \item web app 需支持 Chrome,Firefox,IE11(或更高版本) 等主流浏览器。
    \end{itemize}

\section{技术限制}
% <This subsection could include limitations on the use of specific technologies, interfaces, databases, parallel operations; communications protocols; design conventions or programming standards. >

% 本节包括对使用特定技术的限制,包括接口,数据库,并行操作,通讯协议,设计约定,编程规范等。

数据库和第三方组件应尽可能选用开源产品,便于定制修改。
