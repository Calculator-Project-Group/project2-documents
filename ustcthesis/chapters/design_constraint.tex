\chapter{总体设计约束}
% <Describe any items or issues that will limit the options available to the developers. >

% 描述可能限制开发人员选择的事项。
 
\section{标准符合性}
下面列出了在各种约束中所提及的国际标准、规范:
\begin{itemize}
    \item PEP8标准\\
    \url{https://www.python.org/dev/peps/pep-0008/}
    \item OAuth授权标准\\
    \url{https://oauth.net/2/}
\end{itemize}

\section{硬件约束}
% <This subsection could include Requirements for the software to operate inside various hardware constraints, such as timing constraints, memory constraints etc.)

% 本节包括软件在不同的硬件平台运行的需求,如时间相关的约束,内存方面的约束等。

对于 x86 桌面平台客户端,
要求安装包小于 30M,
安装所需空间小于 150M(不包括数据文件),
运行所需内存小于 200M,
保证低配置电脑也能正常运行。

对于移动 Android 平台客户端,
要求安装包小于 20M,
安装所需空间小于 150M,
运行所需内存小于 300M(Android平台的应用相对x86平台,会占用更多的内存),
保证低配置手机也能正常运行。

对于 x86 平台服务端,
要求安装所需空间小于 500M,
运行所需内存小于 1G。

\section{兼容性约束}
为使产品能够在主流的硬件/软件平台上正确地运行,产品需要满足以下约束要求:
\begin{itemize}
    \item Linux平台下的客户端要支持 Ubuntu,Debian,Centos,Arch 等主流发行版
    \item Windows平台下的客户端需支持 Windows 7 及更高
    \item android 平台下的客户端需支持到安卓 4.4 及以上
    \item web app 需支持 Chrome,Firefox,IE11(或更高版本) 等主流浏览器。
    \end{itemize}

\section{技术限制}
% <This subsection could include limitations on the use of specific technologies, interfaces, databases, parallel operations; communications protocols; design conventions or programming standards. >

% 本节包括对使用特定技术的限制,包括接口,数据库,并行操作,通讯协议,设计约定,编程规范等。

数据库和第三方组件应尽可能选用开源产品,便于定制修改。

\begin{itemize}
    \item \textbf{数据库}\\
    考虑到服务的用户数量较为小众,以及本产品的开源、非营利性政策,选用免费的MySQL作为数据库。
    \item \textbf{服务端开发框架}\\
    考虑到开发周期较短,Web应用需求变化较大等因素,我们选用适合快速开发的Django Framework作为服务端的开发框架。
    \item \textbf{并行计算框架}\\
    对于程序中可能需要并行处理的部分,我们采用广泛使用的MPI并行框架来编写代码。
    \item \textbf{编码规范}\\
    开发过程中,应严格遵守Python代码所推荐遵循的PEP8规范。

\end{itemize}